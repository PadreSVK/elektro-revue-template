\documentclass{elektrorevue}
\usepackage[czech]{babel} 
\usepackage[utf8]{inputenc}

\usepackage[newfloat]{minted}
\usepackage{caption}

\newenvironment{code}{\captionsetup{type=listing}}{}
\SetupFloatingEnvironment{listing}{name=Výpis kódu}

\usepackage{enumitem}
\usepackage{svg}
\usepackage[]{changes}

\begin{document}

\setminted{
    frame=single,
    tabsize=4,
    style=lovelace,
    fontsize=\footnotesize,
    breaklines
}

\def\refname{Literatúra}
\pagestyle{empty}

\twocolumn[\begin{@twocolumnfalse}

%% VYPLNIT: Titul a autorské údaje článku v jednosloupcové sazbě
\nazevClanku{Article Name}
\autor{Author Name\(^1\)} 
\instituce{\(^1\)Fakulta elektrotechniky a komunikačních technologií VUT v~Brně}
\email{author.name@vut.cz}
\vspace{20pt}
\end{@twocolumnfalse}]


\abstrakt{
Lorem ipsum dolor sit amet, consectetur adipiscing elit, sed do eiusmod tempor incididunt ut labore et dolore magna aliqua. Ut enim ad minim veniam, quis nostrud exercitation ullamco laboris nisi ut aliquip ex ea commodo consequat. Duis aute irure dolor in reprehenderit in voluptate velit esse cillum dolore eu fugiat nulla pariatur. Excepteur sint occaecat cupidatat non proident, sunt in culpa qui officia deserunt mollit anim id est laborum.
}

\newacro{eps}[EPS]{Encapsulated PostScript}
\newacro{ps}[PS]{PostScript}
\newacro{pdf}[PDF]{Portable Document Format}
\newacro{dvi}[DVI]{DeVice Independent}
\newacro{jpeg}[JPEG]{Joint Photographic Experts Group}
\newacro{png}[PNG]{Portable Network Graphics}


\input{section_Uvod.tex}
\section{Závěr}
V~tomto dokumentu byly popsány specifické požadavky na grafickou a formální stránku článků pro časopis Elektrorevue. Všechny důležité parametry byly rozebrány, aby se pro nikoho nestalo překážkou používání jiného typografického systému. Uživetelům systému \LaTeX~mohou navíc zdrojové kódy tohoto dokumentu posloužit jako kostra pro psaní nového článku.

% \externalbibliography{yes}

% todo bibtex
% \bibliographystyle{czechiso} % We choose the "plain" reference style
% \bibliography{literatura}
% \begin{thebibliography}{1}
% \end{thebibliography}

\begin{thebibliography}{1}
    
%-------
\bibitem{1}
Belfiore J. \emph{Microsoft Edge: Making the web better through more open source collaboration}\/ [online].[cit.\,19.\,4.\,2020]
Dostupné z~URL:
\(<\)\url{https://blogs.windows.com/windowsexperience/2018/12/06/microsoft-edge-making-the-web-better-through-more-open-source-collaboration/#GmSJg4uFjBM5y8Hz.97}\(>\).
%-------

%-------
\bibitem{2}
Belfiore J. \emph{Microsoft Edge: Making the web better through more open source collaboration}\/ [online].[cit.\,19.\,4.\,2020]
Dostupné z~URL:
\(<\)\url{https://blogs.windows.com/windowsexperience/2018/12/06/microsoft-edge-making-the-web-better-through-more-open-source-collaboration/#GmSJg4uFjBM5y8Hz.97}\(>\).
%-------
    
\end{thebibliography}


\appendix

\end{document}
